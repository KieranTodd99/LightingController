\thispagestyle{empty}

%TC:ignore

\section*{\centering Abstract}

This report aims to outline the process and the decisions made through the duration of my ESP32 lighting controller that was created in the Arduino IDE environment. The aim of the project was to make a fully customisable light emitting diode (LED) controller, with two different LED output options, for the use in dioramas and model scenarios. These LED sequences could either be played constantly, or a trigger could be set on the LED profile. These triggers have  parameters that define under what conditions the LEDs would display its LED sequence. The program also has a website front end, that was hosted from the ESP32, that allows the user to create and edit the LED profiles and define the triggers and its parameters.

The program has the ability for expansion to include other types of triggers and outputs. In the planning section of the project, ideas were included to implement servos, motors and sound into a range of the output options. The ideas suggested allowed for moving parts to be displayed via a trigger in a similar way as the LED profiles were activated. Also with the addition of sound, a MP3 file could be triggered when a profile was displayed. On completion, this would allow for the user to create displays like a level crossings with rising barriers, flashing lights and a warning sound preformed when a train passes by. These features did not get implemented however, as they did not fit within the timescale of the project. 


%TC:endignore