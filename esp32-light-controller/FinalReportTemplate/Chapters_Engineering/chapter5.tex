\chapter{Evaluation}
In this chapter of the document, I have reflected upon the initial decisions made,  design plans and the overall impact and outcome of the project. 

\section{Planning}
As I took the agile methodology approach to producing the program, most of the planning phase was spent understanding what was expected of the final product and doing some spike work and experimenting mainly with hardware that I was going to use. On reflection I feel that this phase of the project is exactly what I needed to understand the scope of the project. In comparison, because most of the hardware that was being used in the project was unfamiliar to me and I have not worked with most of it before, doing a traditional waterfall style planning and design phase would not have been as helpful. This was due to a lack of experience on my part with working some of the aspects of the project. So, when it came to the implementation of the project , I would have known what I wanted the outcome to be, however I would have struggled on implementing the features desired to meet the plans specified in the design brief. This would have led to either a design brief that was not detailed enough or an over enthusiastic design document with requirements that could not have been met.

\section {Hardware Used}
Overall, I have been really impressed with the functionality of the ESP32 and the corresponding hardware (temperature sensor, regular sensor, LEDs and Ring Lights). This is due to the simplicity of how they are used and that ninety percent of the time I did not have any problems with them. From past projects, where I have used other microcontrollers and hardware, most of the time numerous hours were wasted getting the hardware to connect properly and for the code to start reading in the values before you can actually start coding with it. With every piece of hardware used on this project, the setup took at most thirty minutes before I could actually start sending/retrieving data to the different types of hardware used. I am usure whether this simplicity came from the libraries I used for the different pieces of hardware, the guides I was using [link to guides] or was lucky with good hardware but overall I am very happy with how the hardware operated with my code.

\section {Requirements}
In the planning stage of the project, the MoSCoW analysis that was recommended by my supervisor was very helpful. It helped me identify exactly what was required for the completion of the project, what extra functionalities I could add beyond that to make the project better and what functionalities were unrealistic. 

In terms of actually meeting these requirements, I am happy that most of the ‘must have’ requirements have been met and accomplished.  I am content that  I have managed to complete some of the ‘could have’ requirements. However I am disappointed that some of the requirements that I expected the website to be able to do, I have not had time to implement (explained in Following the Methodology and Time Management). Overall I have no discerning opinion on my requirements as I have been able to implement some of the requirements that were considered "extras", while some of the expected requirements were not met on the website. In hindsight I should have stopped working on the extra triggers for the profiles and started focusing on the website, I was enjoying coding with the new hardware I had acquired and as I was already working on the triggers section of the code I thought it would not be too much effort to implement the rest. 

\section {Following the Methodology and Time Management}
The Scrum methodology chosen was ideal. It fitted well with how I normally work and when it came to the planning phase of the project, it helped me understand how I was going to complete the objectives I had planned. When it came to implementing the requirements, the daily scrum and the scrum tasks helped keep me working on the first items that were implemented at the start of the project. However, as time crept on I started doing less and less daily scrums and started to slack on the repetitive tasks that kept me on track. This caused daily task to be bundled together into weekly tasks and just doing rough estimations without any proper thought. This ended-up impacting on me significantly near the end of the project as tasks that I thought would take a week took a week and a half, and so did the next task. Therefore, it did not leave me an appropriate amount of time to work on the website, so I did not meet all of the requirements.

In hindsight one of the things that caused this backlog of work near the end of the project was due to daily tasks not being extensive enough. As one of the features of Scrum is to sprint with your task for the day, when I had finished the tasks that I was working on that day’s sprint, I considered that day finished and stopped working when ideally I could have started working on the next day’s tasks to try and get ahead. 

\section {What I Would Do Differently}
If I had to start this project again or had some advice for myself at the beginning of the project. These are the following pieces of advice I would give to myself: 

\subsection{Don’t underestimate the things you do not understand}
One of the biggest mistakes I made during the planning phase was underestimating the HTML website due to doing some basic HTML a couple years previously. Do not estimate that something is going to be easier or harder depending on a limited previous experience. Two of the biggest misconceptions I had when I started this project was that the HTML would not be that complicated at all and connecting the hardware would be very hard and time consuming. Both of these misconceptions ended up being the opposite of what I thought they were going to be. 

\subsection{Just because something is repetitive and boring does not mean it is not vital}
As discussed previously, one of the biggest flaws that happened during the implementation phase of the project was not keeping up to date with my daily scrums and weekly tasks for the Scum methodology. This caused me to be behind on work when I did not realise how far behind I actually was and for me to estimate how long work would take me when I had not made any real estimates.

\subsection{Balance your time and effort equally}
Something that did not take up too much time but is still relevant as I did spend too much time on certain functions trying to perfect them when there was no need. in particular the triggers took up more time than I should have spent on them when I could have relocated that effort into something that was troubling me or that I had not yet started. 

\section{Conclusion}
The aim of the project was to create a fully customisable LED controller for the use in dioramas and model scenarios. All things considered I feel that I have acheived this this goal. This is because the user is able to create a fully functioning, flexible lighting controller.  The final LED display produced by the lighting controller has numerous output possibilities.  The user can easily set the duration of the lighting sequence, how long the colour/brightness is displayed and have mutiple LEDs all displaying at different timescales. The user also has a range of different variable triggers that they can assign to the LEDs so that the profiles will display the lighting sequences when the desired conditions are met.  However, the user is not yet able to create LED profiles via the website. 

\subsection{If I Had More Time}
If I had more time on the project, the first things I would have to implement would be a proper web front end. To do this I would spend more time on realising how to use SPIFFS for the ESP32 and host the website properly without doing it inline so then I could do the more complicated styling with CSS and HTML to make a good looking final website. 

On reflection, if I had additional time on the project to futher enhance the outcome, the biggest functionality I would have liked to implement would be the use of servos, motors and sound. The hardest aspect of this would be implementing sound to the program as it needs extra hardware for it to work (Storage and a speaker) while the servos and motors only require the servos and motors respectively to make them work. However when these functionalities would have been implemented, this would have allowed the user futher possibilities when it came to the final display of the diorama/model. 



